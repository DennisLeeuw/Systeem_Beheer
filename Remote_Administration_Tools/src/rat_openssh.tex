OpenSSH (\url{https://www.openssh.com/} is een opensource software pakket dat zowel een SSH-server als een SSH-client bevat. OpenSSH is ooit (door)ontwikkeld door OpenBSD ontwikkelaars om een volledig opensource SSH-variant te ontwikkelen zonder patenten.

De meeste op Unix-gebaseerde systemen zoals Linux en Mac OS X zijn standaard uitgerust met een ssh-server of kunnen simpel van een ssh-server voorzien worden.

Om SSH-server op een Mac OS X machine aan te zetten verwijzen we graag naar de website van Apple: \url{https://support.apple.com/lt-lt/guide/mac-help/mchlp1066/mac}

Microsoft biedt OpenSSH aan via zijn eigen repository. Op de site van Microsoft staat beschreven hoe je je Windows Machine kan voorzien van OpenSSH: \url{https://learn.microsoft.com/en-us/windows-server/administration/openssh/openssh_install_firstuse}

SSH gebruikt standaard een username en password om in te kunnen loggen, dat is natuurlijk voor remote beheer niet handig. Een andere optie die de OpenSSH server kent is inloggen via een public en private key. De private key blijft natuurlijk geheim en is eigendom van de beheerder die bij de server moet mogen. De public key(s) wordt geplaatst op de server die men wil beheren. Als iemand wil inloggen op de server en de SSH-client aanroept dan zal deze wat informatie versleutelen met de private key. De server ontvangt dit bericht en zal het decrypten met de public key. Lukt dit dan weet de server zeker dat de inlogger de gene is met de private key. Op deze manier hoef je niet elke keer een username en password in te typen maar kan je op de commandline aan de SSH client meegeven welke private key er gebruikt moet worden.


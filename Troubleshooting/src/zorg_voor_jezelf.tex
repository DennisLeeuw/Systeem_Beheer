Vergeet niet om voor jezelf te zorgen. De beste oplossingen komen vaak op de WC. We nemen tijdens het storingszoeken vaak te weinig pauze en als we dan gedwongen worden door onze blaas om naar de WC te gaan komt soms spontaan de oplossing. Dit is een les. We moeten vaker een pauze nemen om helder te kunnen denken.

Als je uren achter elkaar achter een probleem aan hebt gezeten voel je dan niet schuldig om een kwartier pauze te nemen. Ga wandelen, zoek de frisse lucht op, zorg dat je even weg bent van je werkplek (zelfs als er een server down is of het halve bedrijf niet kan werken). Even rust zorgt vaak voor de oplossing en zorgt er zeker voor dat je daarna helderder nadenkt.

Vergeet ook de interne mens niet. Er is niets zo chagrijning makend als honger of dorst. Sla dus geen maaltijden over, denk om je bloedsuikerspiegel. Denk om je koffie, cola of energiedrank, thee mag natuurlijk ook.

Mocht het echt zo'n dag zijn dat alles tegen zit: zet dan een wekker! Zet na elke pauze een wekker voor de volgende pauze en hou je daar aan!


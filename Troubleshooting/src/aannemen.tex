Bij elk probleem:
\begin{itemize}
	\item Noteer datum en tijd
	\item Noteer de naam van de contactpersoon (je eigennaam als je het zelf bent)
	\item Maak notities!
\end{itemize}

Als je zelf een probleem hebt ontdekt:
\begin{itemize}
\item Maak notities
	\begin{itemize}
	\item Als het mogelijk is: Maak een kopie van de error melding (copy/past, screenshot)
	\item Is dit een nieuwe omgeving of heeft het voorheen goed gewerkt?
	\item Als het voorheen goed gewerkt heeft, wat is er veranderd?
	\item Noteer de omstandigheden (wat was je aan het doen, hoe viel het je op)
	\item Noteer andere zaken die met het probleem te maken kunnen hebben (uitvallen lampjes, beeldscherm, etc.)
	\item Is het probleem herhaalbaar?
	\end{itemize}
\end{itemize}

Een probleem kan gebracht worden door iemand anders. In dat geval:
\begin{itemize}
	\item Maak notities
	\item Vraag door bij onduidelijkheden, controleer de voorgaande lijst om te zien of je alle relevante informatie hebt.
	\item Als de persoon klaar is met zijn verhaal, vat het verhaal samen en controlleer of je goed begrepen hebt wat het probleem is.
	\item Vraag hoe je contact kan houden (e-mail, telefoon, face-to-face?)
	\item Vraag of er meerdere mensen met hetzelfde probleem zijn?
\end{itemize}

Documenteer het probleem in een ticketsysteem. Een centraalsysteem om problemen in op te slaan heeft voordelen.
\begin{itemize}
	\item Anderen kunnen je probleem ook zien (helpen)
	\item Anderen kunnen een probleem van een gebruiker/klant overnemen als je ziek bent
	\item Veel voorkomende problemen worden sneller duidelijk een fix kan gedocumenteerd worden
\end{itemize}

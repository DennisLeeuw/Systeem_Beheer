Als we twee getallen van elkaar willen aftrekken dan is ook weer het aftrekken van twee nullen heel simpel:

\[ 0 - 0 = 0 \]

Voor het aftrekken van een nul van \'e\'en is het ook nog eenvoudig:

\[ 1 - 0 = 1 \]

Het wordt moeilijk bij het aftrekken van \'e\'en van nul. We zouden dan uit moeten komen bij -1, maar dat kan in binair niet. Het binaire stelsel kent geen getallen onder nul. We kunnen dus alleen kleine getallen van grote getallen aftrekken zolang de uitkomst niet lager wordt dan 0. We kunnen dus wel 1 aftrekken van 10:

\[ 10 - 01 = 01 \]

Voor het aftrekken van twee enen is de zaak weer simpel:

\[ 1 - 1 = 0 \]

Nu we dit weten kijk dan eens of de volgende aftrekking klopt:

\begin{center}
\begin{tabular}{ r l }
	11001110 & \\
	01000101 & -\\
	\hline
        10001001 & \\
\end{tabular}
\end{center}


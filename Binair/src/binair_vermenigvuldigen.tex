Het vermenigvuldigen van binaire getallen is niets relatief simpel, vandaar dat computers ook zulke goede rekenaars zijn. Vermenigvuldigen is schuiven en optellen. Het is waarschijnlijk het makkelijkst met een voorbeeld:

\begin{center}
\begin{tabular}{ r l }
      11001110 & \\
      01000101 & x\\
\hline
      11001110 & \\
     000000000 & \\
    1100111000 & \\
   00000000000 & \\
  000000000000 & \\
 0000000000000 & \\
11001110000000 & + \\
\hline
11011110000110
\end{tabular}
\end{center}

Voor elk bit uit het getal waarmee we vermenigvuldigen schuift het getal dat vermenigvuldigt wordt op naar links en krijgt er een 0 achter. Heeft het getal waarmee we vermenigvuldigen een 1 staan dan kom het opgeschoven getal er te staan, staat er een 0 van is de opteller allemaal nullen, kortom er gebeurt niets.

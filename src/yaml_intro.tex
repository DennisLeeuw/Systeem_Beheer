YAML is een taal die data serialiseert zodat deze makkelijk begrepen kan worden door mensen. YAML staat voor YAML Ain't Markup Language. Het is dus niet bedoelt als taal zoals HTML, SGML etc. die aangeeft hoe data (tekst) opgemaakt (Markup) moet worden. YAML kan beter beschreven worden als een taal om data een betekenis te geven. Het is een taal voor configuratie bestanden. Het lijkt qua functie op JSON en XML. Het voordeel van YAML is dat het makkelijk leesbaar is voor mensen en eenvoudig in elkaar zit. YAML stamt uit 2001, meer informatie over de beschikbaarheid van YAML kan gevonden worden op de YAML website: \url{yaml.org}.

YAML bestanden hebben de extensie .yaml of.yml.


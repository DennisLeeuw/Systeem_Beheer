Het over de lijn sturen van scherm informatie is natuurlijk een behoorlijke belasting van het netwerk en niet geschikt voor bijvoorbeeld switches of routers die geen grafische interface hebben. Een al heel oude oplossing hiervoor is het gebruik van de CLI (Command Line Interface) om systemen op afstand te beheren. De eerste oplossing hiervoor was het \texttt{telnet} commando. \texttt{telnet} werd gebruikt om routers en unix-servers te beheren. Het programma is echter nooit geschreven met security in gedachten. Alle commando's en de resultaten gaan in tekst over het netwerk en is dus theoretisch af te luisteren. Ook tijdens het inloggen gaat alles via tekst over de lijn. Gebruikersnamen en wachtwoorden zijn dus simpel te sniffen.

Het is dan ook de bedoeling dat \texttt{telnet} niet meer gebruikt wordt. De wel veilige opvolger is \texttt{ssh} wat een afkorting is voor secure shell en dus wel veilig is. \texttt{ssh} maakt gebruik van encryptie waardoor alles geencrypt over het netwerk gaat.

Het \texttt{ssh} protocol is een client-server-protocol. Er moet dus een stuk software draaien op de host waarmee je verbinding wil maken en op het beheerstation moet de ssh-client ge\"installeerd staan. Standaard draait de ssh-server op port 22. De client bouwt dan ook standaard een verbinding op met deze port. Draait de server op een andere port, dan moet dat aan de client verteld worden.


VNC (Virtual Network Computing) is een client/server architectuur. De server draait op de machine waarvan het scherm overgenomen moet worden. De client is de ontvanger van de scherm informatie.

Het VNC-protocol gebruikt blokjes om de scherm informatie over het netwerk te sturen. Als eenmaal het volledige beeld is opgebouwd zullen alleen de blokjes worden overgestuurd waar zich een wijziging heeft voorgedaan. Om een veilige verbinding op te bouwen met VNC is het nodig VNC te tunnelen door een VPN. Als het alleen op het lokale netwerk wordt gebruikt vallen de veiligheidsrisico's over het algemeen mee.

Since versie 2 (2004) wordt het VNC-protocol ook gebruikt door de Apple Remote Desktop (ARD) applicatie. Hiermee is het mogelijk om met de ARD-client Windows of Linux systemen te beheren die een VNS-server hebben of met elke willekeurige VNC-client een Mac OS X systeem te beheren die een ARD-server heeft.


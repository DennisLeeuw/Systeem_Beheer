YAML is een taal die data serialiseert zodat deze makkelijk begrepen kan worden door mensen. YAML staat voor YAML Ain't Markup Language. Het is dus niet bedoelt als taal zoals HTML, SGML etc. die aangeeft hoe data (tekst) opgemaakt (Markup) moet worden. YAML kan beter beschreven worden als een taal om data een betekenis te geven. Het is een taal voor configuratie bestanden. Het lijkt qua functie op JSON en XML. Het voordeel van YAML is dat het makkelijk leesbaar is voor mensen en eenvoudig in elkaar zit. YAML stamt uit 2001, meer informatie over de beschikbaarheid van YAML kan gevonden worden op de YAML website: \url{yaml.org}.

YAML maakt gebruik van key/value pairs. Het programma dat YAML gebruikt in zijn configuratie bepaalt welke keys geldig zijn binnen een YAML bestand en dus ook welke waardes (values) er toegekend kunnen worden aan een key. In dit document hebben we wat keys en values verzonnen om YAML uit te kunnen leggen, maar de genoemde keys en de mogelijke waarden zijn volledig fictief.

YAML bestanden hebben de extensie .yaml of .yml.


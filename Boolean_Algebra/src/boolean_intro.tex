In dit hoofdstuk gaan we een aantal van de hiervoor beschreven componenten gebruiken gebruiken om computer logica op te bouwen. Kortom hoe maken we van elektrische spanningen logische circuits. De booleanse algebra\index{Booleanse algebra}, een tak van de wiskunde, onderscheidt zich van de rest van de wiskunde doordat de booleanse algebra alleen gebruik maakt van true of false, of zoals we in de computer techniek zeggen 1 en 0.

In de computertechniek gebruiken we de transistor als schakelaar. De basis van de transistor is de input. Door deze input te gebruiken schakelen we de transistor en komt er een bepaalde waarde (1 of 0) aan de output.

De gebruikte spanningen in de computer zijn 3,3 Vdc of 5 Vdc en dat is dan een 1. De 0 is staat gelijk aan geen spanning.

De booleanse algebra wordt ook gebruikt bij het programmeren van computers. Dan staan de waarden 1 of 0 vaak voor true of false. Een vergelijking is waar of niet waar. We zullen hier niet al te doen op ingaan. Wel belangrijk is het om te weten dat je een 1 dus kan lezen als waar en een 0 als niet waar.


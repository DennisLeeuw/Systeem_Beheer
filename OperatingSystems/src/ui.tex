Veelal willen we als mensen computers opdrachten geven. Om dit te kunnen doen moeten we instaat zijn om de computer te besturen, via een muis, toetsenbord of touchscreen. Naast deze input devices moeten we weten hoe we de computer een opdracht geven, door een commando in te typen of door op een bepaald icoontje te klikken. Tot slot moet de computer terug geven wat het resultaat is van de handeling, er moet output zijn naar bijvoorbeeld een scherm of een printer.

Dit alles heeft te maken met de user interface (UI). Het is de interface naar de gebruiker, dit in tegenstelling tot de interface naar de programma's (API). De UI gebruikt de API's om tegen de kernel te praten.


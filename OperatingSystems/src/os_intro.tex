Een operating system of in het Nederlands besturingssysteem is een complex geheel van verschillende stukjes software. Er is niet zoiets als \textbf{het} operating system. We spreken wel van Mac OS X, Windows of Linux, maar eigenlijk zijn dit samenraapsels van verschillende stukken software. De taak van het besturingssysteem is het aansturen van de hardware in opdracht van programma's of gebruikers.

Een gebruiker is een persoon, mens, die via een input-device de computer een opdracht geeft om iets uit te voeren. Een programma is een stuk software dat geschreven is door een programmeur en dat bepaalde opdrachten uitvoert in een bepaalde volgorde. Er zijn een aantal specifieke vormen van programma's:
\begin{description}
\item[Applicatie] Een programma dat door een gebruiker gebruikt kan worden zodat de gebruiker niet direct een opdracht hoeft te geven aan het besturingssysteem maar dat kan doen via een grafische interface. Een applicatie heeft vaak een beperkte functionaliteit. Word is een tekstverwerken en Excel een spreadsheet applicatie.
\item[Commando] Een commando heeft dezelfde functie als een applicatie alleen kent een commando geen grafische interface. Een commando werkt op de command line interface (CLI).
\item[Service] Een service is een programma dat op de achtergrond draait en daar bepaalde functies uitvoert zonder dat de gebruiker zich ermee hoeft te bemoeien.
\end{description}


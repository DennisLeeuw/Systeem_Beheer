Met een kopie van de data op dezelfde harddisk heeft dit gevolgens als de harddisk overlijdt. Het is beter om een kopie te maken op een andere harddisk. Zolang als de data binnen dezelfde omgeving blijft heet dit een on-site backup. Kortom als de lokatie afbrand hebben we nog steeds een probleem.

Een oplossing hiervoor is overbrengen van de kopie naar een andere lokatie. Dit heet dan een off-site backup. De vraag reist nu wel hoeverweg moet de backup zijn om veilig te zijn? Om deze vraag te kunnen beantwoorden moeten we ons afvragen waartegen we ons willen beschermen. Is het voldoende om ons te beschermen tegen een brand op een lokatie of moeten we de data beschermen tegen een nucleaire aanval.


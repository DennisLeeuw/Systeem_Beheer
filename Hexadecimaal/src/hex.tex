Het decimale systeem past niet echt lekker bij computers. Door het gebruik van enen en nullen in bits en bytes valt de decimale scheiding niet samen met de binaire scheiding van getallen. Een systeem dat hier beter bij past is het hexadecimale stelsel. De binaire reeks 1, 2, 4, 8, 16, 32, 64 past naadloos in een hexadecimale telling waarbij er geteld wordt van 0 tot f. Zoals de naam hexadecimaal al aangeeft gaat het om een 16-talig stelsel. We tellen in hexadecimaal van 0 tot 9 en daarna van A tot F voordat er een voorloop 1 voor komt, daarna weer van 0 tot 9 en A tot F voordat de voorloop 1 een 2 wordt. Zijn we bij het voorloopgetal aangekomen bij 9 dan tellen we door naar A en zo verder tot F.

Door van 0 tot f te tellen wordt de notatie van binaire getallen kleiner en beter leesbaar. Een lange reeks van enen en nullen wordt al snel onoverzichtelijk.

Door een stelsel te hebben waarbij er geteld wordt van 0 tot 16 is dit gelijk aan 4 bits in een binair stelsel.
Als we een nibble (4-bits) optellen (decimaal) komen we tot
\begin{math}
2^3 + 2^2 + 2^1 + 2^0 = 8 + 4 + 2 + 1 = 15
\end{math}
. We tellen daarmee van 0 tot 15 en dat is 16-talig. Wat we kunnen weergeven als 0-9 en A tot F.

4-bits heet een nibble. Toch werken we vaker met bytes, oftewel 8-bits. We hebben dus twee hexadecimale getallen nodig om een byte weer te geven, bijvoorbeel EA. Binnen een byte tellen we van 00 naar FF hexadecimaal wat decimaal overeen komt met 0 tot 255.

